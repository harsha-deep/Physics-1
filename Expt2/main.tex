\documentclass{article}
\usepackage[utf8]{inputenc}
\usepackage[super]{nth}
\usepackage[document]{ragged2e}
\usepackage{relsize}
\setlength{\tabcolsep}{4pt}
\usepackage{graphicx} % \scalebox
\usepackage{environ}
\usepackage{amsmath}
\NewEnviron{myequation}{%
\begin{equation}
\scalebox{1.4}{$\BODY$}
\end{equation}
}
% \renewcommand{\arraystretch}{1.5}
% \setlength{\arrayrulewidth}{0.5mm}




\begin{document}
\begin{sloppypar}
\begin{titlepage}
\begin{center}
\vspace*{0.5cm}
\LARGE{\textbf{Physics -1 Lab SM203P}}\\
\Large{\textbf{Experiment - 2}}\\
\vfill
\line(1,0){400}\\[1mm]
\vspace{10mm}
\Large{\textbf{Experiment 2: Experiments with simple and double pendulums}}
\vspace{10mm}
\line(1,0){400}\\[1mm]
\vfill
Group 24 \\
IMT2020065 Shridhar Sharma  \\
IMT2020553 Abhinav Mahajan  \\
IMT2020539 Shaurya Agrawal \\
IMT2020085 Harshadeep Donapati \\
IMT2020126 Ayushmaan Singh \\
IMT2020057 Vishnutha Sheela \\
\end{center}
\end{titlepage}

\section{Aim}
Experiments with simple and double pendulums.
\section{Apparatus used}
\begin{enumerate}
\item A thread
\item PhyPhox application on a mobile device
\item 
\end{enumerate}


\end{sloppypar}
\end{document}

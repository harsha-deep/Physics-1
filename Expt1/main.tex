\documentclass{article}
\usepackage[utf8]{inputenc}
\usepackage[super]{nth}
\usepackage[document]{ragged2e}
\usepackage{relsize}
\setlength{\tabcolsep}{4pt}
\usepackage{graphicx} % \scalebox
\usepackage{environ}
\usepackage{amsmath}
\NewEnviron{myequation}{%
\begin{equation}
\scalebox{1.4}{$\BODY$}
\end{equation}
}
% \renewcommand{\arraystretch}{1.5}
% \setlength{\arrayrulewidth}{0.5mm}




\begin{document}
\begin{sloppypar}
\begin{titlepage}
\begin{center}
\vspace*{0.5cm}
\LARGE{\textbf{Physics -1 Lab SM203P}}\\
\Large{\textbf{Experiment - 1}}\\
\vfill
\line(1,0){400}\\[1mm]
\vspace{10mm}
\Large{\textbf{Experiment 1: Calculation of the acceleration due to gravity using a bouncing ball \& finding the coefficient of restitution}}
\vspace{10mm}
\line(1,0){400}\\[1mm]
\vfill
Group 24 \\
IMT2020065 Shridhar Sharma  \\
IMT2020553 Abhinav Mahajan  \\
IMT2020539 Shaurya Agrawal \\
IMT2020085 Harshadeep Donapati \\
IMT2020126 Ayushmaan Singh \\
IMT2020057 Vishnutha Sheela \\
\end{center}
\end{titlepage}


\section{Aim}
Calculation of the acceleration due to gravity using a
bouncing ball \& finding the coefficient of restitution.
\section{Equipment}
\begin{itemize}
\item A light tennis ball
\item Phyphox app
\item A Ruler
\end{itemize}

% 1
\noindent\makebox[\textwidth]{
\begin{tabular}{|c|c|c|c|c|c| }
\hline
n & 3 & 3 & 3 & 3 & 3 \\ \hline
 h\textsubscript{0}(m) & $ 1.5089\pm0.0001 $ & $ 1.5355 \pm 0.0001 $ & $ 1.5174 \pm 0.0001 $ & $ 1.4793 \pm 1.4402 $ & $ 1.4402 \pm 0.0001 $ \\ \hline
 h\textsubscript{1}(m) & $ 0.9254 \pm 0.0001 $ & $ 0.9323 \pm 0.0001 $ & $ 0.9288 \pm 0.0001 $ & $ 0.9249 \pm 0.0001 $ & $ 0.9055 \pm 0.0001$ \\ \hline
 e = h\textsubscript{1} / h\textsubscript{0} & $ 0.6133 \pm 0.0001 $ & $ 0.6072 \pm 0.0001 $ & $ 0.6121 \pm 0.0001 $ & $ 0.6252 \pm 0.0001 $ & $ 0.6287 \pm 0.0001 $ \\ \hline
 t\textsubscript{n}\textsuperscript{*}(s) & $ 2.097 \pm 0.001 $ & $ 2.099 \pm 0.001 $ & $ 2.095 \pm 0.001 $ & $ 2.098 \pm 0.001 $ & $ 2.086 \pm 0.001 $ \\ \hline
 g(m/s\textsuperscript{2}) & $ 9.6683 \pm 0.0008 $ & $ 9.6404 \pm 0.0011 $ & $ 9.7065 \pm 0.0009 $ & $ 9.8117 \pm 0.0002 $ & $ 9.7628 \pm 0.0002 $  \\ \hline
\end{tabular}
}
\begin{center}
    IMT2020065 Shridhar Sharma
\end{center}
\vfill

% 2
\noindent\makebox[\textwidth]{
\begin{tabular}{|c|c|c|c|c|c| }
\hline
n & 3 & 3 & 3 & 3 & 3 \\ \hline
 h\textsubscript{0}(m) & $1.005 \pm 0.0001$ & $0.981 \pm 0.0001$ & $1.025 \pm 0.0001$ & $1.01 \pm 0.0001$ & $1.014 \pm 0.0001$ \\ \hline
 h\textsubscript{1}(m) & $0.592 \pm 0.0001$ & $0.583 \pm 0.0001$ & $0.595 \pm 0.0001$ & $0.535 \pm 0.0001$ & $0.592 \pm 0.0001$ \\ \hline
 e = h\textsubscript{1} / h\textsubscript{0} & $0.588 \pm 0.0001$ & $0.594 \pm 0.0001$ & $0.581 \pm 0.0001$ & $0.581 \pm 0.0001$ & $0.584 \pm 0.0001$ \\ \hline
 t\textsubscript{n}\textsuperscript{*}(s) & $1.64 \pm 0.001$ & $1.63 \pm 0.001$ & $1.64 \pm 0.001$ & $1.63 \pm 0.001$ & $1.63 \pm 0.001$ \\ \hline
 g(m/s\textsuperscript{2}) & $9.71 \pm 0.017$ & $9.72 \pm 0.017$ & $9.70 \pm 0.017$ & $9.88 \pm 0.017$ & $9.77 \pm 0.017$  \\ \hline
\end{tabular}
}
\begin{center}
    IMT2020553 Abhinav Mahajan 
\end{center}
\vfill

% 3
\noindent\makebox[\textwidth]{
\begin{tabular}{|c|c|c|c|c|c| }
\hline
n & 3 & 3 & 3 & 3 & 3 \\ \hline
 h\textsubscript{0}(m) & $1.466 \pm 0.0001$ & $1.482 \pm 0.0001$ & $1.488 \pm 0.0001$ & $1.527 \pm 0.0001$ & $1.485 \pm 0.0001$ \\ \hline
 h\textsubscript{1}(m) & $0.842 \pm 0.0001$ & $0.856 \pm 0.0001$ & $0.852 \pm 0.0001$ & $0.874 \pm 0.0001$ & $0.850 \pm 0.0001$ \\ \hline
 e = h\textsubscript{1} / h\textsubscript{0} & $0.574 \pm 0.0001$ & $0.574 \pm 0.0001$ & $0.572 \pm 0.0001$ & $0.572 \pm 0.0001$ & $0.572 \pm 0.0001$ \\ \hline
 t\textsubscript{n}\textsuperscript{*}(s) & $1.94 \pm 0.001$ & $1.96 \pm 0.001$ & $1.94 \pm 0.001$ & $1.97 \pm 0.001$ & $1.950 \pm 0.001$ \\ \hline
 g(m/s\textsuperscript{2}) & $9.71 \pm 0.014$ & $9.73 \pm 0.014$ & $9.74 \pm 0.014$ & $9.71 \pm 0.014$ & $9.71 \pm 0.014$  \\ \hline
\end{tabular}
}
\begin{center}
    IMT2020539 Shaurya Agrawal
\end{center}

% 4
\noindent\makebox[\textwidth]{
\begin{tabular}{|c|c|c|c|c|c| }
\hline
n & 3 & 3 & 3 & 3 & 3 \\ \hline
 h\textsubscript{0}(m) & $1.444 \pm 0.0001$ & $1.573 \pm 0.0001$ & $1.448 \pm 0.0001$ & $1.470 \pm 0.0001$ & $1.450 \pm 0.0001$ \\ \hline
 h\textsubscript{1}(m) & $0.919 \pm  0.0001$ & $0.936 \pm  0.0001$ & $0.931 \pm  0.0001$ & $0.943 \pm  0.0001$ & $0.937 \pm  0.0001$ \\ \hline
 e = h\textsubscript{1} / h\textsubscript{0} & $0.6363 \pm 0.0113$ & $0.5953 \pm 0.0110$ & $0.6436 \pm 0.0114$ & $0.6412 \pm 0.0116 $ & $0.6416 \pm 0.0112$ \\ \hline
 t\textsubscript{n}\textsuperscript{*}(s) & $2.128 \pm 0.001$ & $2.111 \pm 0.001$ & $ 2.157 \pm 0.001$ & $2.150 \pm 0.001$ & $2.127 \pm 0.001$ \\ \hline
 g(m/s\textsuperscript{2}) & $9.61 \pm 0.894$ & $9.41 \pm 0.884$ & $9.58 \pm 0.879$ & $9.73 \pm 0.895$ & $9.69 \pm 0.893$  \\ \hline
\end{tabular}
}
\begin{center}
    IMT2020085 Harshadeep Donapati
\end{center}
\vfill

% 5
\noindent\makebox[\textwidth]{
\begin{tabular}{|c|c|c|c|c|c| }
\hline
n & 3 & 3 & 3 & 3 & 3 \\ \hline
 h\textsubscript{0}(m) & $1 \pm 0.0001$ & $1 \pm 0.0001$ & $1 \pm 0.0001$ & $1.5 \pm 0.0001$ & $1.5 \pm 0.0001$ \\ \hline
 h\textsubscript{1}(m) & $0.558 \pm 0.0001$ & $0.562 \pm 0.0001$ & $0.560 \pm 0.0001$ & $0.541 \pm 0.0001$ & $0.526 \pm 0.0001$ \\ \hline
 e = h\textsubscript{1} / h\textsubscript{0} & $0.558 \pm 0.00016$ & $0.562 \pm 0.00015$ & $0.56 \pm 0.00016$ & $0.5413 \pm 0.0001$ & $0.526 \pm 0.0001$ \\ \hline
 t\textsubscript{n}\textsuperscript{*}(s) & $1.563 \pm 0.003$ & $1.607 \pm 0.003$ & $1.597 \pm 0.003$ & $1.901 \pm 0.003$ & $1.93 \pm 0.003$ \\ \hline
 g(m/s\textsuperscript{2}) & $9.71 \pm 0.027$ & $9.30 \pm 0.024$ & $9.36 \pm 0.025$ & $9.32 \pm 0.023$ & $8.85 \pm 0.023$  \\ \hline
\end{tabular}
}
\begin{center}
    IMT2020126 Ayushmaan Singh 
\end{center}


% 6
\noindent\makebox[\textwidth]{
\begin{tabular}{|c|c|c|c|c|c| }
\hline
n & 3 & 3 & 3 & 3 & 3 \\ \hline
 h\textsubscript{0}(m) & $1.1768 \pm 0.0001$ & $1.1338 \pm 0.0001$ & $1.1625 \pm 0.0001$ & $1.1458 \pm 0.0001$ & $1.1307 \pm 0.0001$ \\ \hline
 h\textsubscript{1}(m) & $0.6868 \pm 0.0001$ & $0.6262 \pm 0.0001$ & $0.6538 \pm 0.0001$ & $0.6441 \pm 0.0001$ & $0.6171 \pm 0.0001$ \\ \hline
 e = h\textsubscript{1} / h\textsubscript{0} & $0.584 \pm 0.0001$ & $0.552 \pm 0.0001$ & $0.570 \pm 0.0001$ & $0.562 \pm 0.0001$ & $0.563 \pm 0.0001$ \\ \hline
 t\textsubscript{n}\textsuperscript{*}(s) & $1.731 \pm 0.001$ & $1.606 \pm 0.001$ & $1.721 \pm 0.001$ & $1.635 \pm 0.001$ & $1.625 \pm 0.001$ \\ \hline
 g(m/s\textsuperscript{2}) & $9.76 \pm 0.0002$ & $9.86 \pm 0.0002$ & $9.74 \pm 0.0002$ & $9.82 \pm 0.0002$ & $9.76 \pm 0.0002$  \\ \hline
\end{tabular}
}
\begin{center}
   IMT2020057 Vishnutha Sheela
\end{center}



\section{Theory}
To calculate g, we first make the following assumptions:
\begin{enumerate}
    \item We will ignore air drag (which is almost valid if the ball is dropped from a low height).
    \item The time of contact between ball and floor is very short compared to the free-fall time.
\end{enumerate}
Under the following assumptions, we can use the relation, \[s = \frac{1}{2}gt^2\] to calculate g. But calculating the time taken for a single bounce will introduce intolerable uncertainty. So we consider the time taken by the ball for n bounces.
The height reached by the ball after every bounce will be lower and lower, so we define the coefficient of restitution of energy in the following way:
\begin{equation} 
	\begin{aligned}[b]
	e & \, = \, \frac{mgh_1}{mgh_o} \\
	  & \, = \, \frac{h_1}{h_o} \label{eq:1}
	\end{aligned}
\end{equation}
Taking $t_1$ as the time between the $1^{st}$ and $2^{nd}$ bounces,
\begin{equation} \label{eq:2}
    \begin{aligned}
        g & \, = \, \frac{2 \cdot h_1}{(t_1 / 2)^2} 
          & \, = \, 8 h_1 / t^2_1
    \end{aligned}
\end{equation}
Using \ref{eq:1},
\begin{equation} \label{eq:3}
    g \, = \, 8 h_o e / t^2_1 
\end{equation}
We define $t^*_n$ as the time between the $1^{st}$ and $n^{th}$ bounce,
\begin{equation}
    \begin{aligned}[b]
        t^*_n & \, = \, t_1 + t_2 + \ldots + t_n \\
              & \, = \, t_1 + 2\sqrt{2 h_2 / g} + \ldots + 2\sqrt{2 h_n / g} \\
              & \, = \, t_1 + 2(t_1/2)(\sqrt{e}) + \ldots + 2(t_1/2)(\sqrt{e})^{n - 1} \\
              & \, = \, t_1[1 + \sqrt{e} + \ldots + (\sqrt{e})^{n - 1}] \\
              & \, = \, t_1\frac{(\sqrt{e})^n - 1}{\sqrt{e} - 1} \label{eq:4}
    \end{aligned}
\end{equation}
From \ref{eq:4},
\begin{equation}
         t_1 \,= \, t^*_n \left[\frac{\sqrt{e} - 1} {(\sqrt{e})^n - 1}\right]
\end{equation}
Replacing into \ref{eq:3} we obtain, 

\begin{myequation}
        g \, = \, \frac{8 \cdot h_o \cdot e} {(t^*_n)^2} \cdot \left[\frac{(\sqrt{e})^n - 1} {\sqrt{e} - 1} \right]^2 
\end{myequation}








\section{Error Propogation}
\begin{gather*}
    Relative\ Error\ (\Delta error) = \frac{absolute\ error}{measurement} * 100\%\\
    \\
    \Delta h_0 = \frac{error\ in\ h_0}{measured\ h_0} * 100\%\ ,\ \ \Delta h_1 = \frac{error\ in\ h_1}{measured\ h_1} *100\%\\
    \\
    e = \frac{h_0}{h_1}\\
\end{gather*}
\centering Dividing or multiplying two quantities results in a relative error equal to the sum of their respective relative errors
\begin{gather*}
    \frac{\Delta e}{e}= \frac{\Delta h_0}{h_0} + \frac{\Delta h_1}{h_1}\\
    \frac{\Delta g}{g}=\left[\frac{\Delta h_1}{h_1} + \frac{\Delta h_0}{h_0}\right]\cdot \left[\frac{1+ne^{\frac{n+1}{2}}-(n+1)e^{\frac{n}{2}}}{[1 - \sqrt{e}][1-(\sqrt{e})^n]}\right] + 2\frac{\Delta t_n}{t_n}+\frac{\Delta h_0}{h_0}
\end{gather*}




\section{Sources of error}
\begin{itemize}
\item Air Drag
\item 
\item 
\end{itemize}
\end{sloppypar}
\end{document}
